\documentclass{article}

% if you need to pass options to natbib, use, e.g.:
%     \PassOptionsToPackage{numbers, compress}{natbib}
% before loading neurips_2020

% ready for submission
% \usepackage{neurips_2020}

% to compile a preprint version, e.g., for submission to arXiv, add add the
% [preprint] option:
\usepackage[preprint]{neurips_2020}

% to compile a camera-ready version, add the [final] option, e.g.:
%     \usepackage[final]{neurips_2020}

% to avoid loading the natbib package, add option nonatbib:
%      \usepackage[nonatbib]{neurips_2020}

\usepackage[utf8]{inputenc} % allow utf-8 input
\usepackage[T1]{fontenc}    % use 8-bit T1 fonts
\usepackage{hyperref}       % hyperlinks
\usepackage{url}            % simple URL typesetting
\usepackage{booktabs}       % professional-quality tables
\usepackage{amsfonts}       % blackboard math symbols
\usepackage{nicefrac}       % compact symbols for 1/2, etc.
\usepackage{microtype}      % microtypography
\usepackage{verbatim}

\title{GPU-accelerated Algorithms on solving Stochastic Shortest Path Problems\\ Proposal}

% The \author macro works with any number of authors. There are two commands
% used to separate the names and addresses of multiple authors: \And and \AND.
%
% Using \And between authors leaves it to LaTeX to determine where to break the
% lines. Using \AND forces a line break at that point. So, if LaTeX puts 3 of 4
% authors names on the first line, and the last on the second line, try using
% \AND instead of \And before the third author name.

%  David S.~Hippocampus\thanks{Use footnote for providing further information
 %   about author (webpage, alternative address)---\emph{not} for acknowledging
 %   funding agencies.} \\
  %Department of Computer Science\\
  %Cranberry-Lemon University\\
  %Pittsburgh, PA 15213 \\
  %\texttt{hippo@cs.cranberry-lemon.edu} \\


\author{
  Duo Zhou\\
  Department of Industrial \& Enterprise Engineering\\
  University of Illinois Urbana-Champaign\\
  Urbana, IL 61801 \\
  \texttt{duozhou2@illinois.edu} \\
  % examples of more authors
  \And
  Yilan Jiang \\
  Department of Industrial \& Enterprise Engineering\\
  University of Illinois Urbana-Champaign\\
  Urbana, IL 61801 \\
  \texttt{yilanj2@illinois.edu} \\
  \And
  Tharunkumar Amirthalingam\\
  Department of Industrial \& Enterprise Engineering\\
  University of Illinois Urbana-Champaign\\
  Urbana, IL 61801 \\
  \texttt{ta9@illinois.edu} \\
  % \And
  % Coauthor \\
  % Affiliation \\
  % Address \\
  % \texttt{email} \\
  % \And
  % Coauthor \\
  % Affiliation \\
  % Address \\
  % \texttt{email} \\
}

\begin{document}

\maketitle

\begin{abstract}
  The stochastic shortest path problem is a special version of general shortest path problem combining with 
  a Markov decision process. In this paper, we propose a parallel version of the stochastic shortest path algorithm. 
  The stochastic shortest path problem is a significant area of academic research in the field of optimization. 
  In this paper, we plan that not only would propose a parallel version of the stochastic shortest path algorithm 
  depending on CUDA, but we would also provide a detailed analysis of its performance and characteristics. 
  Our research aims to contribute to the advancement of the field by accelerating the time efficiency of algorithms 
  on stochastic shortest path and exploring its potential uses in Big Data contexts. 
\end{abstract}

\section{Introduction}

Stochastic shortest path is a problem in which we need to find the shortest path from
a given start node to a goal node in a graph, where the edge weights are not deterministic 
but instead are random variables. This problem is commonly used in many applications, 
such as robotics, transportation, and network routing.

Regular shortest path algorithms, such as Dijkstra's algorithm or the A* algorithm, 
assume that the edge weights in a graph are deterministic and known in advance. 
These algorithms work well when the edge weights are fixed and do not change over time. 
However, in many real-world applications, the edge weights are not fixed but instead are 
subject to randomness and uncertainty. For example, in transportation networks, the travel 
time between two locations can vary depending on traffic conditions, weather, accidents, etc. 
Similarly, in robotic navigation, the cost of moving from one location to another can depend 
on sensor readings, terrain features, obstacles, etc. In such scenarios, regular shortest path 
algorithms may not be appropriate as they do not take into account the randomness and uncertainty 
of the edge weights. Stochastic shortest path algorithms, on the other hand, explicitly model the 
probabilistic nature of the edge weights and aim to find the path with the lowest expected cost.

The stochastic shortest path problem is a challenging problem in stochastic optimization, 
and its solution requires a combination of mathematical and computational techniques. 
The need for using GPUs, particularly CUDA, to speed up the stochastic shortest path algorithm 
depends on various factors such as the size of the graph, the complexity of the edge weight distributions, 
and the available computing resources. In general, stochastic shortest path algorithms involve computations 
with matrices and vectors, which can be computationally intensive for large graphs. GPUs, 
with their massively parallel architecture and high memory bandwidth, can accelerate these 
computations significantly, leading to faster computation times. Moreover, if the edge weight distributions 
are complex and involve high-dimensional probability distributions, such as multivariate normal or mixture distributions, 
then the computations involved in the algorithm may be even more demanding. In such cases, 
GPUs can provide significant speedups compared to CPU-based implementations.

\section{Literature Review}

There have been plenty of algorithms innovated to resolve the general shortest path problem in the past decades. 
Gallo and Pallottino discussed eight algorithms for the shortest path problem and compared their performances 
with various data structures [1]. To further increase the computational efficiency, Crauser et al\. proposed a 
parallelized version of the Dijkstra's algorithm [2]. Even though the shortest path problem can be applied to 
many areas like road networks and social media, it is less generic to be applied to some other real life problems 
such as robotics, autonomous driving, and network routing.  

Bertsekas and Tsitsiklis generalized the shortest path problem and first introduced the idea of stochastic shortest path (SSP) 
problem in 1991 [3]. They further extended the corresponding theory of Markovian decision problems 
by removing the assumptions that cost can either be all positive or all negative. Building on top 
of that, Polychronopoulos and Tsitsiklis developed a dynamic programming algorithm to resolve a SSP 
problem to devise a policy that leads from an origin to a destination with minimal expected cost (4). 
Further works have been done to expand the ecosystem of SSP problem by proposing new concepts and frameworks [5, 6]. 
In addition, SSP is defined as an instance of a Markov Decision Process. 

On Markov Decision Processes with Parallel Algorithm, Archibald, T. W. et al. looked at serial value iteration algorithms for Markov decision processes and develops efficient parallel alternatives [7]. 
Jóhannsson, Á.Þ. et al. introduced two CUDA-based implementations of the Value Iteration algorithm: Block Divided Iteration and Result Divided Iteration 
and showed a substantial performance improvement for the parallel algorithms compared to a sequential implementation 
on a CPU [12]. Ruiz and Hernandez formulated a MDP solver based on the Value Iteration algorithm that uses matrix 
multiplications. This allows us to leverage GPUs to produce interactive obstacle-free paths in the form of an Optimal Policy [10]. 

Ortega-Arranz, H. et al. presented a GPU SSSP algorithm implementation  significantly sped up the computation of the SSSP. They also enhanced the GPU 
algorithm implementation using proper choice of threadblock size and the modification of the GPU L1 cache memory state of 
NVIDIA devices [11]. Sapio et al. introduced two new MDP solvers for embedded 
systems: Sparse Value Iteration (SVI), which operates on small, single-threaded CPU platforms using sparse matrix methods, 
and Sparse Parallel Value Iteration (SPVI), which takes advantage of embedded graphics processing units (GPUs) to increase 
performance on more advanced embedded systems [8]. Steimle et al. developed exact and fast approximation methods with error bounds 
for Multi-model Markov decision process (MMDP), which generates a single policy maximizing performance over all models. 
This approach allows the decision maker to trade-off conflicting data sources while creating a policy of the same complexity 
for models that consider only one data source [9].

\section{Proposed work}

\subsection{Tasks}

In Task 1, we will spend sometime to explore the mathematical theories as well as algorithms 
relevant to stochastic shortest path and markov decision processes in greater details. 
By diving into these topics, we will be able to better understand the underlying principles 
that govern these processes and apply this knowledge to real-world situations. 

For Task 2, in order to solve the stochastic shortest path problem, we can reproduce various 
existing algorithms. These algorithms can be implemented using a CPU as a baseline scenario. 
By utilizing these techniques, we can efficiently identify the optimal path in a given graph, 
even when the graph has stochasticity. Additionally, we may consider implementing these algorithms 
on GPUs especially using CUDA in order to further optimize the solution.

\subsection{Data}

We will evaluate our method on a list of infinite-horizon Multi-model Markov Decision Processes (MMDPs). 
These MMDPs are test instances generated by using the codes provided by University of Michigan-Deep Blue Data. 
We will use their codes to generate three sets of MMDPs, each corresponding to different medical decision-making problems. 
These problems include the initiation of HIV therapy, a machine maintenance problem, and a set of randomly constructed instances. 
A more detailed description of the data under each problem can be found online. 

\begin{center}
  \url{https://deepblue.lib.umich.edu/data/concern/data_sets/t722h899p}
\end{center}

\subsection{Milestons}

Referring to the schedule Table~\ref{Milestons}, our milestone is shown above.


\begin{table}
  \caption{Milestons}
  \label{Milestons}
  \centering
  \begin{tabular}{lll}
    \toprule
    %\multicolumn{2}{c}{Part}                   \\
    \cmidrule(r){1-2}
    Time     & Tasks \\
    \midrule
    Mar. 9 & Proposal Investigation    \\
    Mar. 23 & Review of Literature \& Algorithm Design   \\
    Mar. 28 & Finish up Miderm Report  \\
    Apr. 13 & Mathematical proof of Algorithm  \\
    Apr. 23 & Experimental Design \& Validation \\
    Apr. 25 & Final Report Completion \\
    \bottomrule
  \end{tabular}
\end{table}

\subsection{Testing and Expected results}

We will choose CUDA enabled NVIDIA GPU and C++ to realize the paralleled computations and use python 
to take care of all other programming works including data preprocessing and pipeline construction. 
We will test our method compared to the baseline model using CPU counterparts in terms of speed. 
It is important to choose the appropriate method or algorithm for the specific problem at hand and to 
carefully analyze the results to ensure that the solution is accurate and effective. 

\begin{comment}
\section{Submission of papers to NeurIPS 2020}

NeurIPS requires electronic submissions.  The electronic submission site is
\begin{center}
  \url{https://cmt3.research.microsoft.com/NeurIPS2020/}
\end{center}

Please read the instructions below carefully and follow them faithfully.

\subsection{Style}

Papers to be submitted to NeurIPS 2020 must be prepared according to the
instructions presented here. Papers may only be up to eight pages long,
including figures. Additional pages \emph{containing only a section on the broader impact, acknowledgments and/or cited references} are allowed. Papers that exceed eight pages of content will not be reviewed, or in any other way considered for
presentation at the conference.

The margins in 2020 are the same as those in 2007, which allow for $\sim$$15\%$
more words in the paper compared to earlier years.

Authors are required to use the NeurIPS \LaTeX{} style files obtainable at the
NeurIPS website as indicated below. Please make sure you use the current files
and not previous versions. Tweaking the style files may be grounds for
rejection.

\subsection{Retrieval of style files}

The style files for NeurIPS and other conference information are available on
the World Wide Web at
\begin{center}
  \url{http://www.neurips.cc/}
\end{center}
The file \verb+neurips_2020.pdf+ contains these instructions and illustrates the
various formatting requirements your NeurIPS paper must satisfy.

The only supported style file for NeurIPS 2020 is \verb+neurips_2020.sty+,
rewritten for \LaTeXe{}.  \textbf{Previous style files for \LaTeX{} 2.09,
  Microsoft Word, and RTF are no longer supported!}

The \LaTeX{} style file contains three optional arguments: \verb+final+, which
creates a camera-ready copy, \verb+preprint+, which creates a preprint for
submission to, e.g., arXiv, and \verb+nonatbib+, which will not load the
\verb+natbib+ package for you in case of package clash.

\paragraph{Preprint option}
If you wish to post a preprint of your work online, e.g., on arXiv, using the
NeurIPS style, please use the \verb+preprint+ option. This will create a
nonanonymized version of your work with the text ``Preprint. Work in progress.''
in the footer. This version may be distributed as you see fit. Please \textbf{do
  not} use the \verb+final+ option, which should \textbf{only} be used for
papers accepted to NeurIPS.

At submission time, please omit the \verb+final+ and \verb+preprint+
options. This will anonymize your submission and add line numbers to aid
review. Please do \emph{not} refer to these line numbers in your paper as they
will be removed during generation of camera-ready copies.

The file \verb+neurips_2020.tex+ may be used as a ``shell'' for writing your
paper. All you have to do is replace the author, title, abstract, and text of
the paper with your own.

The formatting instructions contained in these style files are summarized in
Sections \ref{gen_inst}, \ref{headings}, and \ref{others} below.

\section{General formatting instructions}
\label{gen_inst}

The text must be confined within a rectangle 5.5~inches (33~picas) wide and
9~inches (54~picas) long. The left margin is 1.5~inch (9~picas).  Use 10~point
type with a vertical spacing (leading) of 11~points.  Times New Roman is the
preferred typeface throughout, and will be selected for you by default.
Paragraphs are separated by \nicefrac{1}{2}~line space (5.5 points), with no
indentation.

The paper title should be 17~point, initial caps/lower case, bold, centered
between two horizontal rules. The top rule should be 4~points thick and the
bottom rule should be 1~point thick. Allow \nicefrac{1}{4}~inch space above and
below the title to rules. All pages should start at 1~inch (6~picas) from the
top of the page.

For the final version, authors' names are set in boldface, and each name is
centered above the corresponding address. The lead author's name is to be listed
first (left-most), and the co-authors' names (if different address) are set to
follow. If there is only one co-author, list both author and co-author side by
side.

Please pay special attention to the instructions in Section \ref{others}
regarding figures, tables, acknowledgments, and references.

\section{Headings: first level}
\label{headings}

All headings should be lower case (except for first word and proper nouns),
flush left, and bold.

First-level headings should be in 12-point type.

\subsection{Headings: second level}

Second-level headings should be in 10-point type.

\subsubsection{Headings: third level}

Third-level headings should be in 10-point type.

\paragraph{Paragraphs}

There is also a \verb+\paragraph+ command available, which sets the heading in
bold, flush left, and inline with the text, with the heading followed by 1\,em
of space.

\section{Citations, figures, tables, references}
\label{others}

These instructions apply to everyone.

\subsection{Citations within the text}

The \verb+natbib+ package will be loaded for you by default.  Citations may be
author/year or numeric, as long as you maintain internal consistency.  As to the
format of the references themselves, any style is acceptable as long as it is
used consistently.

The documentation for \verb+natbib+ may be found at
\begin{center}
  \url{http://mirrors.ctan.org/macros/latex/contrib/natbib/natnotes.pdf}
\end{center}
Of note is the command \verb+\citet+, which produces citations appropriate for
use in inline text.  For example,
\begin{verbatim}
   \citet{hasselmo} investigated\dots
\end{verbatim}
produces
\begin{quote}
  Hasselmo, et al.\ (1995) investigated\dots
\end{quote}

If you wish to load the \verb+natbib+ package with options, you may add the
following before loading the \verb+neurips_2020+ package:
\begin{verbatim}
   \PassOptionsToPackage{options}{natbib}
\end{verbatim}

If \verb+natbib+ clashes with another package you load, you can add the optional
argument \verb+nonatbib+ when loading the style file:
\begin{verbatim}
   \usepackage[nonatbib]{neurips_2020}
\end{verbatim}

As submission is double blind, refer to your own published work in the third
person. That is, use ``In the previous work of Jones et al.\ [4],'' not ``In our
previous work [4].'' If you cite your other papers that are not widely available
(e.g., a journal paper under review), use anonymous author names in the
citation, e.g., an author of the form ``A.\ Anonymous.''

\subsection{Footnotes}

Footnotes should be used sparingly.  If you do require a footnote, indicate
footnotes with a number\footnote{Sample of the first footnote.} in the
text. Place the footnotes at the bottom of the page on which they appear.
Precede the footnote with a horizontal rule of 2~inches (12~picas).

Note that footnotes are properly typeset \emph{after} punctuation
marks.\footnote{As in this example.}

\subsection{Figures}

\begin{figure}
  \centering
  \fbox{\rule[-.5cm]{0cm}{4cm} \rule[-.5cm]{4cm}{0cm}}
  \caption{Sample figure caption.}
\end{figure}

All artwork must be neat, clean, and legible. Lines should be dark enough for
purposes of reproduction. The figure number and caption always appear after the
figure. Place one line space before the figure caption and one line space after
the figure. The figure caption should be lower case (except for first word and
proper nouns); figures are numbered consecutively.

You may use color figures.  However, it is best for the figure captions and the
paper body to be legible if the paper is printed in either black/white or in
color.

\subsection{Tables}

All tables must be centered, neat, clean and legible.  The table number and
title always appear before the table.  See Table~\ref{sample-table}.

Place one line space before the table title, one line space after the
table title, and one line space after the table. The table title must
be lower case (except for first word and proper nouns); tables are
numbered consecutively.

Note that publication-quality tables \emph{do not contain vertical rules.} We
strongly suggest the use of the \verb+booktabs+ package, which allows for
typesetting high-quality, professional tables:
\begin{center}
  \url{https://www.ctan.org/pkg/booktabs}
\end{center}
This package was used to typeset Table~\ref{sample-table}.

\begin{table}
  \caption{Sample table title}
  \label{sample-table}
  \centering
  \begin{tabular}{lll}
    \toprule
    \multicolumn{2}{c}{Part}                   \\
    \cmidrule(r){1-2}
    Name     & Description     & Size ($\mu$m) \\
    \midrule
    Dendrite & Input terminal  & $\sim$100     \\
    Axon     & Output terminal & $\sim$10      \\
    Soma     & Cell body       & up to $10^6$  \\
    \bottomrule
  \end{tabular}
\end{table}

\section{Final instructions}

Do not change any aspects of the formatting parameters in the style files.  In
particular, do not modify the width or length of the rectangle the text should
fit into, and do not change font sizes (except perhaps in the
\textbf{References} section; see below). Please note that pages should be
numbered.

\section{Preparing PDF files}

Please prepare submission files with paper size ``US Letter,'' and not, for
example, ``A4.''

Fonts were the main cause of problems in the past years. Your PDF file must only
contain Type 1 or Embedded TrueType fonts. Here are a few instructions to
achieve this.

\begin{itemize}

\item You should directly generate PDF files using \verb+pdflatex+.

\item You can check which fonts a PDF files uses.  In Acrobat Reader, select the
  menu Files$>$Document Properties$>$Fonts and select Show All Fonts. You can
  also use the program \verb+pdffonts+ which comes with \verb+xpdf+ and is
  available out-of-the-box on most Linux machines.

\item The IEEE has recommendations for generating PDF files whose fonts are also
  acceptable for NeurIPS. Please see
  \url{http://www.emfield.org/icuwb2010/downloads/IEEE-PDF-SpecV32.pdf}

\item \verb+xfig+ "patterned" shapes are implemented with bitmap fonts.  Use
  "solid" shapes instead.

\item The \verb+\bbold+ package almost always uses bitmap fonts.  You should use
  the equivalent AMS Fonts:
\begin{verbatim}
   \usepackage{amsfonts}
\end{verbatim}
followed by, e.g., \verb+\mathbb{R}+, \verb+\mathbb{N}+, or \verb+\mathbb{C}+
for $\mathbb{R}$, $\mathbb{N}$ or $\mathbb{C}$.  You can also use the following
workaround for reals, natural and complex:
\begin{verbatim}
   \newcommand{\RR}{I\!\!R} %real numbers
   \newcommand{\Nat}{I\!\!N} %natural numbers
   \newcommand{\CC}{I\!\!\!\!C} %complex numbers
\end{verbatim}
Note that \verb+amsfonts+ is automatically loaded by the \verb+amssymb+ package.

\end{itemize}

If your file contains type 3 fonts or non embedded TrueType fonts, we will ask
you to fix it.

\subsection{Margins in \LaTeX{}}

Most of the margin problems come from figures positioned by hand using
\verb+\special+ or other commands. We suggest using the command
\verb+\includegraphics+ from the \verb+graphicx+ package. Always specify the
figure width as a multiple of the line width as in the example below:
\begin{verbatim}
   \usepackage[pdftex]{graphicx} ...
   \includegraphics[width=0.8\linewidth]{myfile.pdf}
\end{verbatim}
See Section 4.4 in the graphics bundle documentation
(\url{http://mirrors.ctan.org/macros/latex/required/graphics/grfguide.pdf})

A number of width problems arise when \LaTeX{} cannot properly hyphenate a
line. Please give LaTeX hyphenation hints using the \verb+\-+ command when
necessary.


\section*{Broader Impact}

Authors are required to include a statement of the broader impact of their work, including its ethical aspects and future societal consequences. 
Authors should discuss both positive and negative outcomes, if any. For instance, authors should discuss a) 
who may benefit from this research, b) who may be put at disadvantage from this research, c) what are the consequences of failure of the system, and d) whether the task/method leverages
biases in the data. If authors believe this is not applicable to them, authors can simply state this.

Use unnumbered first level headings for this section, which should go at the end of the paper. {\bf Note that this section does not count towards the eight pages of content that are allowed.}

\begin{ack}
Use unnumbered first level headings for the acknowledgments. All acknowledgments
go at the end of the paper before the list of references. Moreover, you are required to declare 
funding (financial activities supporting the submitted work) and competing interests (related financial activities outside the submitted work). 
More information about this disclosure can be found at: \url{https://neurips.cc/Conferences/2020/PaperInformation/FundingDisclosure}.


Do {\bf not} include this section in the anonymized submission, only in the final paper. You can use the \texttt{ack} environment provided in the style file to autmoatically hide this section in the anonymized submission.
\end{ack}

\end{comment} 

\section*{References}
%\ref{reference}
\small

[1] Gallo, G.,\ \& Pallottino, S.\ (1988). Shortest path algorithms. {\it Annals 
  of operations research} {\bf 13}(1): 1-79.

[2] Crauser, A., Mehlhorn, K., Meyer, U.,\ \& Sanders, P. (1998). A paralleliztion 
  of Dijkstra's shortest path algorithm. {\it In Mathematical Foundatio
  s of Computer Science 1998: 23rd International Symposium MFCS'98 Brno, 
  Czech Republic, August 24-28, 1998 Proceedings} {\bf 23}(pp. 722-731). Springer Berlin Heidelberg.

[3] Bertsekas, D. P.,\ \& Tsitsiklis, J. N.\ (1991). An analysis of stochastic shortest path problems. 
  {\it Mathematics of Operations Research} {\bf 16}(3): 580-595.

[4] Polychronopoulos, G. H.,\ \& Tsitsiklis, J. N.\ (1996). Stochastic shortest path problems with 
  recourse. {\it Networks: An International Journal} {\bf 27}(2): 133-143.

[5] Ji, X.\ (2005). Models and algorithm for stochastic shortest path problem. {\it Applied Mathematics 
  and Computation} {\bf 170}(1): 503-514.

[6] Guillot, M.,\ \& Stauffer, G.\ (2020). The stochastic shortest path problem: a polyhedral combinatorics 
  perspective. {\it European Journal of Operational Research} {\bf 285}(1): 148-158.

[7] Archibald, T. W., McKinnon, K. I. M.,\ \& Thomas, L. C.\ (1993). Serial and parallel value iteration 
  algorithms for discounted Markov decision processes. {\it European journal of operational research} {\bf 67}(2): 188-203.

[8] Sapio, A., Bhattacharyya, S. S.,\ \& Wolf, M.\ (2020, July). Efficient model solving for Markov decision processes. 
  {\it In 2020 IEEE Symposium on Computers and Communications (ISCC)}(pp. 1-5). IEEE.

[9] Steimle, L. N., Kaufman, D. L.,\ \& Denton, B. T.\ (2021). Multi-model Markov decision processes. 
  {\it IISE Transactions} {\bf 53}(10): 1124-1139.

[10] Ruiz, S.,\ \& Hernández, B.\ (2015, October). A parallel solver for Markov decision process in crowd simulations. 
  {\it In 2015 Fourteenth Mexican International Conference on Artificial Intelligence (MICAI)}(pp. 107-116). IEEE.

[11] Ortega-Arranz, H., Torres, Y., Gonzalez-Escribano, A.,\ \& Llanos, D. R.\ (2015). Comprehensive evaluation of 
  a new GPU-based approach to the shortest path problem. {\it International Journal of Parallel Programming} {\bf 43}(5), 918-938.

[12] Jóhannsson, Á.Þ.\ (2009). GPU-based Markov decision process solver.

\end{document}